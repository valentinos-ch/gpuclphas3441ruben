\documentclass[a4paper]{article}
\usepackage{natbib}

\begin{document}
\title{Nuclear Disarmament Verification}
\author{Aldridge, Barritt, Christodoulou, Lu, \\
	Moors, Rajroop, Theodorou, Xiao, \\
	Department of Physics and Astronomy, \\
	University College London, \\
	London WC1E 6BT, \\
	United Kingdom}
\date{2012 March 06\textsuperscript{th}}
\maketitle

\section*{Executive summary}
\subsection*{The Context}
  Why do we Verify?
We verify to assure confidence. We have a declaration, our verification methods, and our confidence. If we are too intrusive with our verification methods there is the threat of the spread of proliferation information which could in turn lead to a breakdown of the declaration agreements alongside an increased risk of mass devastation. If the verification methods employed are too feeble, the uncertainty in the confidence will be high and inspectors may fail to detect anomalous behaviour in the host’s disarmament activities.
This balance is known as the information barrier and it is spawn from the consideration of ratified treaties, the capabilities of the equipment and the feasibility of the procedure. The treaties outline what information should and shouldn’t be shared with the inspectors; the equipment and procedure are the technological or economic constraints.
*A chart of time, treaty and declaration information or disarmament rates, there should be an inflection at 2002 *
Summer’s treaty review
Nick’s information barrier in the context of treaty
Luke’s weapon introduction
Jack’s confidence in verification
\subsection*{The Process}
	What do we verify?
Luke’s weapon introduction
Nick’s brief overview 
Ralph’s dismantling process
Valentino’s chain of custody / Containment and surveillance of dismantlement 
Luke’s Blending down
\subsection*{The Methods}
	How do we verify?
Jenelle’s Passive detection; Gamma signature, Neutron signature
Kaijian’s detection methods; Induced Gamma and Neutron signature 
Nick’s Pit stuffing
Luke’s blending down
\subsection*{The Technology}
	What do we use to verify and how does it work?
Valentino’s Containment and surveillance / Ralph’s review of tags and seals tech
Kaijian’s detection methods
Jenelle’s Passive detection
Valentino’s Muon Tomography 
\subsection*{The Conclusion}
	(What are the Strengths weaknesses opportunities and threats?)
\subsection*{Final Summary}

\tableofcontents

\section{Introduction}
\subsection{Sample Section}
Here is a sample statement, of which I shall 
include the most well known equation of all. 
Indeed, the irony is that this equation is 
intimately related to our group project. 
\[ 
E = \gamma m_0 c^2
\]

\section{Political Background}
\subsection{Treaties}

\end{document}
